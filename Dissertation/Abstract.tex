\documentclass[a4paper]{article}

\usepackage{fullpage} 
\usepackage{parskip} 
\usepackage{tikz}
\usepackage{graphicx}
\usepackage{caption}
\usepackage{subcaption}
\usepackage{float}
\usepackage{amsmath}
\usepackage{hyperref}
\usepackage{multirow}
\usepackage{wrapfig}
\usepackage{lipsum}

\title{MSc Dissertation - Draft 1 \\~\\ Modeling Heterogeneity in Microbial Population Dynamics}
\author{Helena Herrmann}
\date{\today}


\begin{document}

\maketitle

\section*{Abstract}

\textbf{Chapter 1} 
The rate at which microbial cells progress through the cell cycle is a major component of their evolutionary fitness.
Measuring fitness phenotypes in a given environment or genetic background forms the basis of most quantitative assays of drug sensitivity or genetic interaction, for example, including genome-wide assays.
Growth rate is typically measured in bulk cell populations, inoculated with anything from hundreds to millions of cells sampled from purified, isogneic colonies.
High-throughput microscopy reveals that surprising levels of growth rate heterogeneity arise between isogenic cell lineages \cite{Levy12,Ziv13}.
Here we examine the implications for interpreting bulk, population scale growth rate observations, given observed levels of growth rate heterogeneity.
We demonstrate that selection between cell lineages with a range of growth rates can give rise to an apparent lag phase at the population level, even in the absence of evidence for any lag phase at the lineage level.
Given observed levels of heterogeneity, we predict the effect of inoculum density on growth rate estimates and their reproducibility in bulk population \textit{S. cerevisiae} growth experiments.

\textbf{Chapter 2}
In order to validate and further explore conclusions from Chapter 1, we carried out high-throughput microscopy experiments on Quantitative Fitness Analysis (QFA) \textit{S. cerevisiae} cultures.
We call this approach ($\mu$QFA).
We designed image analysis tools for $\mu$QFA to ensure that capture growth curves for fast-growing outlier lineages as well as for non-dividing cells.
Fast-growing outliers in particular influence the extent of the lag phase at the population level.
We explored various options for modelling growth curves and for carrying out parameter inference for models, including the full workflow in an open source R package.
{SHOULD PROBABLY ONLY DISCUSS THE BEST COMBINATION OF MODEL AND INFERENCE WORKFLOW IN THE ABSTRACT?}

\section{Isogenic growth rate heterogeneity and how population-level observations fail to capture it.}

\section{$\mu$QFA data confirms a population lag phase as a result of growth rate heterogeneity and is used to explore available growth rate inference techniques.}

\begin{thebibliography}{}
\bibitem[1]{Levy12} Levy,S.F., Ziv,N., Siegal,M.L. (2012) Bet hedging in yeast by heterogeneous, age-correlated expression of a stress protectant. {\it PLoS Biol.}, {\bf 10}, e1001325.
\bibitem[2]{Ziv13}Ziv,N., Siegal,M.L., Gresham,D. (2013) Genetic and nongenetic determinants of cell growth variation assessed by high-throughput microscopy. {\it Mol. Biol. Evol.}, {\bf 30}, 2568–2578. 

\end{thebibliography}{}

\end{document}
